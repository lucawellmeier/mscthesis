\chapter{Modern frontiers in statistical learning theory}

\section{Probabilistic model}
\begin{itemize}
    \item product space model, loss, problem formulation 
        \cite{smale_learning}
    \item review of conditional probabilities, distributions 
        and expectation \cite{billingsley,dudley,royden}
    \item regression function, squared loss (and others) and 
        noise \cite{smale_learning}
    \item discussion of squared loss \cite{fitwithoutfear}
    \item hypothesis space and true risk 
        \cite{smale_learning}
    \item marginal and conditional variant of
        fubini \cite{dudley,billingsley}
\end{itemize}

\section{Statistical learning and regression problems}
\begin{itemize}
    \item samples and empirical measure
        \footnote{\url{https://en.wikipedia.org/wiki/Empirical_measure}}
    \item empirical risk
    \item consistency \cite{vapnik}
    \item decomposition into approximation and sample error 
        \cite{smale_learning}
    \item generalization and structural risk minimization
        \cite{vapnik}
\end{itemize}

\section{New frontiers: the interpolating regime}
\begin{itemize}
    \item the classical bias-variance trade-off and SRM point of 
        view on generalization \cite{vapnik}
    \item they don't explain highly over-parameterized deep learning
        \cite{fitwithoutfear,understandkernel,rethink}
    \item double descent \cite{reconciling,twomodels}
    \item interpolation is not always bad also for other estimators; 
        benign overfitting theoretically \cite{benignlinear,benignridge}
        and practically \cite{justinterpolate,surprises}
    \item distinctions into catastrophic, tempered and benign overfitting
        with kernel ridgeless regression examples \cite{taxonomy}
    \item but where is generalization coming from if not from 
        explicit regularization? 
        \cite{benignimplicit,justinterpolate,deepimplicitreg}
\end{itemize}

\section{Sobolev kernels as study model}
\begin{itemize}
    \item most of these phenomena are captured by kernel methods
        \cite{justinterpolate,understandkernel}
    \item in particular laplace kernel is of interest: shares RKHS with
        neural tangent kernel \cite{ntk,ntklaplace}
    \item more generally we will study matern kernels coming from sobolev
        spaces: advantages include smoothness control, easy interpolation,
        and very expressive hypothesis spaces
    \item already shown to be only consistent in high dimensions 
        \cite{laplaceconsistency,sobolevconsistency}
\end{itemize}