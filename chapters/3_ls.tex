\chapter{Search algorithm: Least squares}

% TODO: add example of well-posed finite-dimensional linear regression
%       and how to obtain the estimator

% TODO: talk about ill-posed linear inverse problems and the 
%       abundance of least-squares solutions

%%%%%%%%%%%%%%%%%%%%%%%%%%%%%%%%%%%%%%%%%%%%%%%%%%%%%%%%%%%%%%%%%%%%%%%%%%%%%%
\section{Least squares solutions for linear inverse problems}
%%%%%%%%%%%%%%%%%%%%%%%%%%%%%%%%%%%%%%%%%%%%%%%%%%%%%%%%%%%%%%%%%%%%%%%%%%%%%%

% literature
We follow \cite[Chapter 3]{clason}, for this section.

% well-posed linear inverse problems
Let $H_1$ and $H_2$ be fixed Hilbert spaces and consider a linear
operator $A \colon H_1 \to H_2$ between them.
A \emph{linear inverse problem} consists of solving the problem
\begin{equation*}
    A x = y
\end{equation*}
in $x \in H_1$ for a given $y \in H_2$.
Such a problem is called \emph{well-posed} if three conditions 
are satisfied:
a solution $x \in H_1$ exists, it is unique and the dependence
of $x$ on $y$ is continuous (if $Ax_n \to y$ then $x_n \to x$).
For instance, in finite dimensional Euclidean space, well-posedness
is equivalent to $\det A \neq 0$ where the matrix inverse is 
a continous linear operator.

% ill-posed problems
However, most interesting cases we will encounter are 
\emph{ill-posed} or "barely well-posed" in the sense of high 
condition numbers.
We will develop theory here to solve them approximately
by finding $x \in H_1$ that minimizes the 
distance $\|Ax - y\|_{H_2}$ instead of attempting to find true 
solutions.
We will see later that even if solving for least squares, 
it is not guaranteed to have solutions.
However, if this is the case we tendentially have infinitely 
many, which is when we choose the one with minimal norm.

\begin{defn}
    Let $A \in B(H_1,H_2)$ be a bounded operator.
    An element $x^\dagger \in H_1$ is called a least squares 
    (ls) solution of the inverse problem $Ax = y$ if 
    \begin{equation*}
        \| Ax^\dagger - y \|_{H_2} 
        = \min_{z \in X} \| A z - y \|_{H_2}.
    \end{equation*}
    If $x^\dagger$ is a least squares solution and has minimal
    norm among all other, we call it a minimum-norm least 
    squares (mnls) solution.
\end{defn}

% pseudoinverse
In the following we construct an operator, the \emph{pseudoinverse}, 
whose core is defined on the domain of true invertibility and then 
extended to the maximal domain in which ls solutions exist.
Concretely, it maps any $y$ in its domain to the corresponding mnls 
solution.
\begin{prop}[Construction of pseudoinverse]
    Let $A \in B(H_1,H_2)$ be a bounded linear operator.
    Set 
    \begin{equation*}
        \widetilde A \coloneqq A|_{\ker(A)^\perp}
            \colon \ker(A)^\perp \to \ran(A).
    \end{equation*}
    Then, $\widetilde A^{-1}$ extends uniquely to a linear operator
    $A^\dagger$ such that 
    \begin{equation*}
        \dom(A^\dagger) = \ran(A) \oplus \ran(A)^\perp, \quad
        \ker(A^\dagger) = \ran(A)^\perp.
    \end{equation*}
\end{prop}
\begin{proof}
    $\widetilde A$ is bijective and linear, so that $A^\dagger$ exists 
    on $\ran(A)$ and is linear there.
    Now let $y \in \dom(A^\dagger)$ be uniquely decomposed as 
    $y = y_1 + y_2$ with $y_1 \in \ran(A)$ and $y_2 \in \ran(A)^\perp$ 
    according to the orthogonal sum.
    We must have $A^\dagger y_2 = 0$ since 
    $y_2 \in \ker(A^\dagger) = \ran(A)^\perp$.
    Thus, defining
    \begin{equation*}
        A^\dagger y \coloneqq A^\dagger y_1 + A^\dagger y_2 
        = A^\dagger y_1 = \widetilde{A}^{-1} y_1
    \end{equation*}
    yields the only possible linear extension.
\end{proof}

\begin{lemma}[Properties of pseudoinverses]
\label{lemma:pinv_props}
    Let $A \in B(H_1, H_2)$, 
    and let $P_k$ and $P_r$ be the orthogonal projections 
    onto $\ker(A)$ and $\overline{\ran}(A)$, respectively.
    Then:
    \begin{enumerate}
        \item \label{p:pinv_range} 
            $\ran(A^\dagger) = \ker(A)^\perp$,
        \item \label{p:pinv_projker}
            $A^\dagger A = I - P_k$ and $A A ^\dagger A = A$,
        \item \label{p:pinv_projran}
            $AA^\dagger = P_r|_{\dom(A^\dagger)}$
            and $A^\dagger A A^\dagger = A^\dagger$.
    \end{enumerate}
\end{prop}
\begin{proof}
    \begin{enumerate}
        \item For all 
            $y \in \dom(A^\dagger) = \ran(A) \oplus \ran(A)^\perp$ 
            we have $P_r y \in \ran(A)$ (and not only in its closure!) 
            by the direct sum decomposition so that by construction 
            $A^\dagger y = A^\dagger P_r y = \widetilde A^{-1} P_r y$. 
            Therefore, 
            $A^\dagger y \in \ran(\widetilde A^{-1}) = \ker(A)^\perp$
            and $\ran(A^\dagger) \subseteq \ker(A)^\perp$.
            Conversely, if $x \in \ker(A)^\perp$, then 
            $A^\dagger A x = \widetilde A^{-1} \widetilde A x = x$ 
            so that $\ker(A)^\perp \subseteq \ran(A^\dagger)$.
    
        \item If $x \in H_1$,
            \begin{align*}
                A^\dagger A x &= \widetilde A^{-1} A x 
                    = \widetilde A^{-1} A (P_k x + (I - P_k)x) \\
                    &= \underbrace{\widetilde A^{-1} A P_k x}_{= 0}
                        + \underbrace{\widetilde A^{-1} \widetilde A}_{= I}
                        (I - P_k) x
                    = (I - P_k) x.
            \end{align*}
            Hence,
            \[ A A^\dagger A = A (I - P_k) = A - A P_k = A. \]
        
        \item If $y \in \dom(A^\dagger)$, then by the same arguments 
            as in \cref{p:pinv_range}
            \begin{equation*}
                A A^\dagger y = A \widetilde A^{-1} P_r y 
                = \widetilde A \widetilde A^{-1} P_r y = P_r y.
            \end{equation*}
            Moreover,
            \begin{equation*}
                A^\dagger A A^\dagger y = A^\dagger P_r y 
                = \widetilde A^{-1} y = A^\dagger y.
            \end{equation*}
    \end{enumerate}
\end{proof}

Finally, we are going to see that the pseudoinverse is exactly the 
right object to compute for finding mnls solutions as claimed in 
the beginning.
\begin{thm}[Characterization of ls solutions]
\label{thm:char_ls_solution}
    If $y \in \dom(A^\dagger)$, then the problem 
    $\min_{x \in H_1} \| y - Ax \|$ has solutions 
    which, in turn, are exactly the solutions of the equation
    \begin{equation} \label{eq:char_ls_projran}
        Ax = P_r y.
    \end{equation}
    Among these, there is a unique solution $x^\dagger$ of minimum 
    norm given by $x^\dagger = A^\dagger y$ and the set of solutions 
    to the least squares problem is $x^\dagger + \ker(A)$.
\end{thm}
\begin{proof}
    Since $P_r y \in \ran(A)$ there is at least one solution $z$ of 
    \cref{eq:char_ls_projran}.
    By the optimality of orthogonal projections we have
    \begin{equation*}
        \|y-Az\| = \|y - P_r y\| 
        = \min_{w \in \overline{\ran}(A)} \|y-w\| \leq \|y - Ax\|
    \end{equation*}
    for all $x \in H_1$.
    On the other hand, if $z$ is a least squares solution then
    \begin{equation*}
        \|y-P_r y\| \leq \|y - Az\| = \min_{x \in H_1} \|y-Ax\|
        = \min_{w \in \ran(A)} \|w - y\|
        \leq \|y-P_r y\|
    \end{equation*}
    so we have an equality.
    Hence the least squares solutions are exactly those to 
    \cref{eq:char_ls_projran}.
    
    Each least squares solution $x$ can be decomposed as 
    $x = \overline x + x_0 \in \ker(A)^\perp \oplus \ker(A)$.
    Suppose there was another solution $x' = \overline x' + x_0'$. 
    Then, $A \overline x = A x = P_r y = A x' = A \overline x'$
    but $A$ is injective on $\ker(A)^\perp$, so 
    $\overline x = \overline x'$ depends only on $x$.
    By orthogonality of the two components
    \begin{equation*}
        \| x \|^2 = \|\overline x + x_0\|^2 
        = \|\overline x\|^2 + \|x_0\|^2 \geq \|\overline x\|^2,
    \end{equation*}
    which implies
    \begin{equation*}
        x^\dagger \coloneqq \overline x = P_k \overline x 
        = (I - P_k) \overline x = A^\dagger A \overline x 
        = A^\dagger P_r y = A^\dagger A A^\dagger y 
        = A^\dagger y
    \end{equation*}
    is the unique mnls solution and 
    (we have freely used the properties in \cref{lemma:pinv_props}).
\end{proof}

% normal equations, ls and mnls solutions
The previous characterization states that $x \in H_1$ is a 
ls solution iff $Ax = P_r y$, that is, iff
$Ax \in \overline\ran(A)$ (empty condition) and 
$Ax - y \in (\overline\ran(A))^\perp$.
Taking into account that 
$(\overline\ran(A))^\perp = \ker(A^*)$,
the second condition means that ls solutions to $Ax = y$ are 
exactly those that solve the \emph{normal equation}
$A^*(Ax - y) = 0$ usually presented as
\begin{equation}\label{eq:normal}
    A^* A x = A^* y.
\end{equation}
Moreover, this implies that the mnls solution of $Ax=y$ is 
the solution to \cref{eq:normal} of minimal norm, hence by  
the previous theorem
\begin{equation}\label{eq:normal_mnls}
    x^\dagger = (A^* A)^\dagger A^* y.
\end{equation}

% reduction to selfadjoint
Of particular importance is the fact that \cref{eq:normal_mnls}
reduces the computation of the mnls solution to the pseudoinverse
of the selfadjoint operator $(A^*A)$.
This gives us access to spectral methods that will be explored
in \cref{section:spectral_pinv}.

% continuity
However, for now we are going to tackle the question of the 
third condition of well-posed problems: continuous dependence
of the solution on the data $y$.
We will see that this is closely linked to the maximal domain
of the pseudoinversed operator.
Observe that
\begin{equation*}
    \overline{\dom(A^\dagger)}
    = \overline{\ran(A) \oplus \ran(A)^\perp}
    = \overline\ran(A) \oplus \ran(A)^\perp
    = \ker(A^*)^\perp \oplus \ker(A^*)
    = H_2,
\end{equation*}
which means that the domain is always dense in $H^2$.
A crucial point of the following will be the question of 
whether $y$ is contained in $\overline\ran(A)$ or even
in $\ran(A)$.
In fact, if $\ran(A)$ is closed, then the maximal domain 
is the whole space and we will see that $A^\dagger$ is a 
bounded, hence continuous, operator in this case.
We will also see that this is both sufficient and necessary
for the continuity.

\begin{thm}[Continuity of pseudoinverse]
\label{thm:pinv_continuity}
    Let $A \in B(H_1,H_2)$.
    Then the pseudoinverse 
    $A^\dagger \colon \dom(A^\dagger) \to H_1$ is
    continuous if and only if $\ran(A)$ is closed.
\end{thm}
\begin{proof}
    \begin{itemize}
        \item Let us first suppose that $T^\dagger$ is bounded,
            which implies that it maps Cauchy sequences to Cauchy 
            sequences.
            Since $\dom(A^\dagger)$ is dense in $H^2$ we can 
            pick a sequence $y_n \to y$ contained in the domain 
            for a given $y$ and define a continuous extension
            $\overline{A^\dagger} y \coloneqq \lim A^\dagger y_n$.
            If we pick $y \in \overline\ran(A)$ and 
            $(y_n) \subset \ran(A)$ we have by \cref{lemma:pinv_props}
            \begin{equation*}
                y = P_r y = \lim P_r y_n = \lim A A^\dagger y_n
                    = A \overline{A^\dagger} y \in \ran(A),
            \end{equation*}
            and thus, $\overline\ran(A) = \ran(A)$.
        \item Let now, conversely $\overline\ran(A) = \ran(A)$
            i.e. $\dom(A^\dagger) = H_2$.
            In order to prove that $A^\dagger \colon H_2 \to H_1$ is 
            continuous, we invoke the Closed Graph theorem stating 
            that a linear operator between Banach spaces 
            (note that the assumption enters here for the 
            completeness)
            is continuous iff it has a closed graph.
            In particular, we suppose that we have a sequence
            $y_n \to y$ in $H_2$ so that $A^\dagger y_n \to x \in H_1$ 
            and we have to show that $A^\dagger y = x$.
            With the help of \cref{lemma:pinv_props} and the 
            continuity of $A$ we find 
            \begin{equation*}
                P_r y = \lim P_r y_n = \lim AA^\dagger y_n 
                = A A^\dagger y = A x,
            \end{equation*}
            which means that $x$ is a ls solution of $Ax = y$, 
            e.g. of the form $x = x^\dagger + x_0$ with 
            $x_0 \in \ker(A)$.
            Finally, since $\ran(A^\dagger) = \ker(A)^\perp$ is 
            closed, 
            we have $A^\dagger y_n \to x \in \ker(A)^\perp$,
            so that $x_0$ must be 0 and thus
            $A^\dagger y = x^\dagger = x$.
    \end{itemize}
\end{proof}

% TODO: add note about the compact infinite range

%%%%%%%%%%%%%%%%%%%%%%%%%%%%%%%%%%%%%%%%%%%%%%%%%%%%%%%%%%%%%%%%%%%%%%%%%%%%%%
\section{Pseudoinverse via singular value decomposition}
\label{section:spectral_pinv}
%%%%%%%%%%%%%%%%%%%%%%%%%%%%%%%%%%%%%%%%%%%%%%%%%%%%%%%%%%%%%%%%%%%%%%%%%%%%%%

% compact selfadjoint
As already hinted around \cref{eq:normal_mnls}, 
we put special attention to the case of compact 
selfadjoint operators where we always have an eigensystem.
The interesting operators $A$ in our settings
will be compact and we already now that computing
their pseudoinverses can be reduced to computing 
$(A^* A)^\dagger$, which is selfadjoint.

% spectral theorem
We recall the spectral theorem for compact selfadjoint operators.
The proof can be found in any standard text like TODO.
\begin{thm}[Spectral theorem for compact operators]
\label{thm:spectral}
    Consider a compact and selfadjoint operator $K \in K(H,H)$ on 
    a Hilbert space $H$.
    Then, there exists an orthonormal system $(e_n)_{n\in\N}$ in $H$ 
    and a sequence $(\lambda_n)_{n \in \N} \to 0$ of non-negative 
    reals (both possibly finitely null-terminating) such that 
    \begin{equation*}
        K x = \sum_n \lambda_n \, \langle x,e_n \rangle e_n,
    \end{equation*}
    The orthonormal system forms a basis of $\overline\ran(A)$.
\end{thm}
Clearly, the $e_n$ are eigenvectors since $K e_n = \lambda_n e_n$.
We call $( \lambda_n, e_n )_n$ an \emph{eigensystem} of $K$,
where we always assume that $\lambda_1 \geq \lambda_2 \geq \dots$.

% svd
This can be generalized to not necessarily selfadjoint but 
compact operators using the \emph{singular value decomposition}
(svd). 
\begin{thm}[Singular value decomposition]
    Let $K \in K(H_1,H_2)$ be a compact operator between two 
    Hilbert spaces.
    There is a non-increasing null-sequence $(\sigma_n)_{n \in \N}$ 
    of positive numbers and 
    onbs $(u_n)_n$ and $(v_n)_n$ of $\overline\ran(K)$ and 
    $\overline\ran(K^*)$, respectively, all possibly finitely
    null-terminating, such that
    $K v_n = \sigma_n u_n$, $K^* u_n = \sigma_n v_n$ and 
    \begin{equation}\label{eq:svd}
        K x = \sum_n \sigma_n \langle x, v_n \rangle_{H_1} u_n.
    \end{equation}
\end{thm}
\begin{proof}
    We just give the constructions for the sake of better 
    understanding but skip all checks.
    Details can be found in \cite[Theorem 3.9]{clason}.

    The operator $K^* K$ is selfadjoint and compact again, so 
    that it admits an eigensystem $(\lambda_n, v_n)_n$
    by \cref{thm:spectral}.
    The $v_n$ are as required if we also consider 
    $\sigma_n \coloneqq \sqrt{\lambda_n}$ 
    (or $\sigma_n \coloneqq 0$ if the $v_n$ sequence terminates).
    Finally, we define 
    $u_n \coloneqq \sigma_n^{-1} K v_n$.
\end{proof}
As short-hand notation we use $(\sigma_n,u_n,v_n)_n$ to denote a
\emph{singular system}.
Note that the eigenvalues of $K^*K$ with eigenvectors $v_n$ 
are exactly those $KK^*$ with eigenvectors $u_n$.
Hence, we can also write
\begin{equation*}
    K^* y = \sum_n \sigma_n \langle y, u_n \rangle_{H_2} v_n.
\end{equation*}

% back to pseudoinverse
The pseudoinverse of a compact operator can be conveniently 
computed when its svd is known in the most natural way thinkable:
by inverting non-zero singular values.
Following up on \cref{thm:pinv_continuity} and the discussion 
above, such an expression is only possible on the "continuous"
part of the maximal domain.
The next result characterizes this using the so-called 
\emph{Picard condition}.

\begin{thm}[Picard conditions]
\label{thm:pinv_picard}
    Let $K \in K(H_1,H_2)$ be a compact operator with 
    singular system $(\sigma_n, u_n, v_n)_n$.
    Let $y \in \overline\ran(K)$.
    Then, $y \in \ran(K)$ iff
    \begin{equation} \label{eq:picard}
        \sum_{n} \sigma_n^{-2} 
        \, |\langle y,u_n \rangle_{H_2}|^2 < \infty.
    \end{equation}
    In that case the pseudoinverse can be written as
    \begin{equation} \label{eq:pinv_svd}
        K^\dagger y = \sum_n \sigma_n^{-1} 
        \langle y,u_n \rangle_{H_2} v_n.
    \end{equation}
\end{thm}
\begin{proof}
    \begin{itemize}
        \item If $y \in \ran(K)$, there is $x \in H_1$ 
            such that $Kx = y$.
            By the properties of singular values
            \begin{equation*}
                \langle y, u_n \rangle_{H_2} 
                = \langle x, K^* u_n \rangle_{H_1}
                = \sigma_n \langle x, v_n \rangle_{H_1}.
            \end{equation*}
            Hence, by Bessels inequality:
            \begin{equation*}
                \sum_n \sigma_n^{-2}
                    \,|\langle_{H_2} y,u_n \rangle|^2
                = \sum_n |\langle x, v_n \rangle_{H_1}|^2.
            \end{equation*}
        \item Conversely, if $y \in \overline\ran(K)$ satisfies 
            the Picard condition, the partial sums of the rhs
            of \cref{eq:picard} form a Cauchy sequence.
            Then, also the sequence defined by 
            \begin{equation*}
                x_N = \sum_{n=1}^N \sigma_n^{-1} 
                \langle y,u_n\rangle_{H_2} v_n
            \end{equation*}
            is a Cauchy sequence since $\|x_N - x_M\|^2$ is easily 
            bounded by the former.
            It convergences to the series $x$ in the closed space
            $\overline\ran(K^*) = \ker(K)^\perp$.
            Therefore, using the fact that the $u_n$ are an onb 
            for $\overline\ran(K)$,
            \begin{equation*}
                Kx = \sum_n \sigma_n^{-1} 
                    \langle y,u_n \rangle_{H_2} K v_n
                = \sum_n \langle y,u_n \rangle_{H_2} u_n
                = P_r y = y,
            \end{equation*}
            i.e. $y \in \ran(K)$.
    \end{itemize}
    \Cref{eq:pinv_svd} follows from the fact that $Kx = P_r y$
    is equivalent to $x = K^\dagger y$ for $x \in \ker(K)^\perp$.
\end{proof}

%%%%%%%%%%%%%%%%%%%%%%%%%%%%%%%%%%%%%%%%%%%%%%%%%%%%%%%%%%%%%%%%%%%%%%%%%%%%%%
\section{Learning with least squares: Linear regression}
%%%%%%%%%%%%%%%%%%%%%%%%%%%%%%%%%%%%%%%%%%%%%%%%%%%%%%%%%%%%%%%%%%%%%%%%%%%%%%

%%%%%%%%%%%%%%%%%%%%%%%%%%%%%%%%%%%%%%%%%%%%%%%%%%%%%%%%%%%%%%%%%%%%%%%%%%%%%%
\section{Spectral regularization: ridge regression}
%%%%%%%%%%%%%%%%%%%%%%%%%%%%%%%%%%%%%%%%%%%%%%%%%%%%%%%%%%%%%%%%%%%%%%%%%%%%%%